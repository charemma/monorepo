\documentclass[a4paper,10pt]{article}
\usepackage[utf8]{inputenc}
\usepackage[T1]{fontenc}
\usepackage[german]{babel}
\usepackage{geometry}
\usepackage{fontspec}
\usepackage[german]{babel}
\usepackage{hyperref}
\usepackage{parskip}
\usepackage{enumitem}
\usepackage{xcolor}
\usepackage{fancyhdr}
\usepackage{titlesec}
\geometry{a4paper, margin=1in}
\usepackage{lastpage}

% Define color
\definecolor{maincolor}{RGB}{38, 139, 210} % charemma.de solarized.dark #268BD2 blue
\definecolor{background}{RGB}{0, 0, 0} % charemma.de solarized.light #fff white
\definecolor{textcolor}{RGB}{64, 64, 64} % charemma.de solarized.light #939393 grey

% Set font to Lato Light
\setmainfont{Lato Light}[
    BoldFont = Lato Bold,
    ItalicFont = Lato Light Italic,
    BoldItalicFont = Lato Bold Italic
]

% Set section and subsection formatting
\titleformat{\section}{\large\bfseries\color{maincolor}}{}{0em}{}[]
\titleformat{\subsection}{\normalsize\bfseries\color{maincolor}}{}{0em}{}

% Set text color
\color{textcolor}

\pagestyle{fancy}
\fancyhf{}
\fancyfoot[L]{\textcolor{maincolor}{\thepage}}
%\fancyfoot[L]{\textcolor{maincolor}{\thepage} / \textcolor{maincolor}{\pageref{LastPage}}}
% Kontakt jetzt im Header

% Zeichne kein rechteck um hyperlink adressen, z.B. in https://xxxx
\hypersetup{
    colorlinks=true,
    linkcolor=maincolor,
    urlcolor=maincolor,
    citecolor=maincolor
}

\begin{document}

\begin{center}
    {\Huge\color{textcolor}charalambos} \\
    \vspace{0.1cm}
    {\Huge\textbf{\color{maincolor}EMMANOUILIDIS}} \\
    \vspace{0.3cm}
    {\normalsize Stuttgart | +49 177 8734354 | \href{mailto:me@charemma.de}{me@charemma.de} | \url{charemma.de}}
\end{center}

\section*{Profil}
Platform Engineer mit Fokus auf Developer Experience und CI/CD. Ich baue interne Entwicklungsplattformen (Coder, DevContainers), automatisiere Build-Prozesse und kümmere mich um Infrastruktur als Code. Hintergrund in Embedded Linux (Yocto) und IT-Security.

\vspace{0.5cm}
\section*{Kenntnisse}
\begin{itemize}[leftmargin=0.5cm]
    \item \textbf{Platform/DevX}: Coder, DevContainers, GitHub Actions, GitLab CI
    \item \textbf{IaC}: Pulumi, Ansible, Terraform
    \item \textbf{Container}: Docker, Kubernetes, k3s
    \item \textbf{Embedded}: Yocto, Buildroot, Bitbake
    \item \textbf{Sprachen}: Python, Golang, Bash, C++
    \item \textbf{Security}: OSCP Training, Bug Bounty, OWASP
\end{itemize}

\section*{Freiberufliche Tätigkeit}

\subsection*{Platform Engineering, Philips Medizin Systeme Böblingen}
\textit{September 2024 - heute}
\begin{itemize}[leftmargin=0.5cm]
    \item Build-Optimierung im Monorepo: Performance-Analyse, sparse-checkout und selektive Builds eingeführt
    \item Justfile-basiertes Tooling für CLI-Workflows (Cross-Build, SDK-Management, QEMU), fehlende Dokumentation ergänzt
    \item Remote-Entwicklungsplattform mit Coder auf k3s (IaC mit Pulumi und Ansible)
    \item DevContainer-Modernisierung: Entkopplung von Yocto-SDK, leichtere Images
    \item DevEx-Consulting: Platform Engineering Operating Model zur Stärkung der Team-Autonomie
\end{itemize}

\vspace{0.5em}

\subsection*{Platform Engineering, ARRI Cine Technik München}
\textit{Juli 2022 - Juni 2024}
\begin{itemize}[leftmargin=0.5cm]
    \item Yocto-Projekt Pflege und LTS-Upgrade für professionelle Kamera-Systeme
    \item Migration von Jenkins zu GitLab CI mit Docker-basierten Runnern
    \item Remote-Build-Umgebung für Entwickler (Docker, VS Code Remote)
    \item Plattformkonzept: Self-Service Provisionierung mit Pulumi (GitLab, Artifactory)
    \item Wiederverwendbare CI-Module mit Dagger für KAS-Builds und QA
    \item Ansible und GitOps für Infrastruktur-Management
\end{itemize}

\vspace{0.5em}

\subsection*{DevOps Engineering, Philips Medizin Systeme Böblingen}
\textit{April 2021 - Juni 2022}
\begin{itemize}[leftmargin=0.5cm]
    \item DevOps-Einführung im Model-Based Systems Engineering (MBSE)
    \item GitHub Actions Pipeline für Java-Plugin Distribution
    \item CLI-Tool für Cameo REST API (Benutzer- und Rechteverwaltung)
    \item Remote-Einarbeitung und Wissenstransfer
\end{itemize}

\section*{Berufserfahrung (Festanstellung)}

\subsection*{Philips Medizin Systeme Böblingen}
\textit{März 2006 - November 2019}

\textbf{DevOps Architect} (2017-2019)
\begin{itemize}[leftmargin=0.5cm]
    \item CI/CD-Pipeline für Yocto-basierte Embedded Builds
    \item Infrastructure as Code Einführung, Microservice-Architektur für Tooling
    \item DevOps-Konzept für regulierte Medizinproduktentwicklung
\end{itemize}

\textbf{DevOps Engineer / Scrum Master} (2014-2017)
\begin{itemize}[leftmargin=0.5cm]
    \item GitLab und Docker Registry Setup, Ansible-Provisionierung
    \item Scrum-Einführung als erster Scrum Master im Bereich
    \item Microservices in Go für Build-Tooling
\end{itemize}

\textbf{Embedded Developer} (2006-2014)
\begin{itemize}[leftmargin=0.5cm]
    \item Yocto-Einführung für Produktentwicklung
    \item C++ Realtime-Applikationen, Python-Tooling
    \item Build-System Entwicklung (Make, CMake)
\end{itemize}

\section*{Studium}
\textbf{Dipl. Ing. Softwaretechnik und Medieninformatik}, Hochschule Esslingen (2006)

\section*{IT-Security}
OSCP Training (Offensive Security), Web Application Security (SCP), Bug Bounty (HackerOne)

\section*{Sprachen}
Deutsch (Muttersprache), Englisch (Fließend), Griechisch (Fließend)

\section*{Links}
\url{https://www.linkedin.com/in/charemma} | \url{https://charemma.de}

\end{document}
