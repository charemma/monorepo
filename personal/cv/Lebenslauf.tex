\documentclass[a4paper,10pt]{article}
\usepackage[utf8]{inputenc}
\usepackage[T1]{fontenc}
\usepackage[german]{babel}
\usepackage{geometry}
\usepackage{fontspec}
\usepackage[german]{babel}
\usepackage{hyperref}
\usepackage{parskip}
\usepackage{enumitem}
\usepackage{xcolor}
\usepackage{fancyhdr}
\usepackage{titlesec}
\geometry{a4paper, margin=1in}
\usepackage{lastpage}

% Define color
\definecolor{maincolor}{RGB}{38, 139, 210} % charemma.de solarized.dark #268BD2 blue
\definecolor{background}{RGB}{0, 0, 0} % charemma.de solarized.light #fff white
\definecolor{textcolor}{RGB}{64, 64, 64} % charemma.de solarized.light #939393 grey

% Set font to Lato Light
\setmainfont{Lato Light}[
    BoldFont = Lato Bold,
    ItalicFont = Lato Light Italic,
    BoldItalicFont = Lato Bold Italic
]

% Set section and subsection formatting
\titleformat{\section}{\large\bfseries\color{maincolor}}{}{0em}{}[]
\titleformat{\subsection}{\normalsize\bfseries\color{maincolor}}{}{0em}{}

% Set text color
\color{textcolor}

\pagestyle{fancy}
\fancyhf{}
\fancyfoot[L]{\textcolor{maincolor}{\thepage}}
%\fancyfoot[L]{\textcolor{maincolor}{\thepage} / \textcolor{maincolor}{\pageref{LastPage}}}
% Kontakt jetzt im Header

% Zeichne kein rechteck um hyperlink adressen, z.B. in https://xxxx
\hypersetup{
    colorlinks=true,
    linkcolor=maincolor,
    urlcolor=maincolor,
    citecolor=maincolor
}

\begin{document}

\begin{center}
    {\Huge\color{textcolor}charalambos} \\
    \vspace{0.1cm}
    {\Huge\textbf{\color{maincolor}EMMANOUILIDIS}} \\
    \vspace{0.3cm}
    {\normalsize Stuttgart | +49 177 8734354 | \href{mailto:me@charemma.de}{me@charemma.de} | \url{charemma.de}}
\end{center}

\section*{Profil}
Platform Engineer und IT-Security Spezialist mit doppelter Expertise: Einerseits Konzeption und Betrieb skalierbarer Entwicklungsplattformen (DevOps, CI/CD), andererseits Security-Audits und Schwachstellenanalyse in komplexen Infrastrukturen. Die Kombination aus tiefem Infrastruktur-Verständnis und Offensive Security ermöglicht einen ganzheitlichen Blick auf sichere Systemarchitekturen.

\vspace{0.5cm}
\section*{Kenntnisse}
\begin{itemize}[leftmargin=0.5cm]
  \item \textbf{Platform Engineering (Architektur \& Konzepte)}: 
        Interne Entwicklerplattformen, Self-Service-Modelle, 
        Platform Operating Models, CI/CD- und Build-Architekturen,
        Monorepo-Strategien, Governance in regulierten Umgebungen

  \item \textbf{Developer Experience (Enablement \& Tooling)}: 
        Coder, DevContainers, VS Code, 
        GitHub Actions, GitLab CI, Dagger, 
        CLI-Tooling (Just, Bash, Python, Go)

  \item \textbf{Infrastructure as Code}: Pulumi, Terraform, Ansible, GitOps

  \item \textbf{Container \& Orchestrierung}: Docker, Kubernetes, k3s

  \item \textbf{Embedded Linux}: Yocto, BitBake, KAS

  \item \textbf{Programmiersprachen}: Python, Go, Bash, C++

  \item \textbf{IT-Security (Offensive Security)}:
      OSCP (Offensive Security Certified Professional),
      OSWE (in progress, seit Feb. 2026),
      Burp Suite, Metasploit, nmap, sqlmap, Wireshark,
      Fuzzing (wfuzz, ffuf), Privilege Escalation,
      Web Application Security, Active Directory Attacks,
      Responsible Disclosure (u.\,a. Bundeswehr),
      Bug Bounty (HackerOne)
\end{itemize}


\section*{Freiberufliche Tätigkeit}

\subsection*{Platform Engineering \& Security, Philips Medizin Systeme Böblingen}
\textit{September 2024 - laufend}
\begin{itemize}[leftmargin=0.5cm]
    \item Security-Audits interner Web-Applikationen und Docker-Container-Infrastruktur:
    Schwachstellenanalyse (Authentication, Input Validation, Container-Security),
    Erstellung detaillierter Hardening-Empfehlungen
    (Base-Image-Auswahl, Least Privilege, Secrets Management)
    \item Analyse und Optimierung eines großen Monorepos:
    Einführung von sparse-checkout, selektiven Builds und Performance-Messungen
    zur Reduktion von Build-Zeiten
    \item Konzeption und Implementierung eines Justfile-basierten CLI-Toolings
    für Cross-Builds, SDK-Management und QEMU-Workflows;
    Konsolidierung und Ergänzung fehlender Dokumentation
    \item Aufbau einer Remote-Entwicklungsplattform mit Coder auf k3s
    unter Berücksichtigung hoher Sicherheitsanforderungen (regulierte Medizintechnik),
    vollständig automatisiert mittels Infrastructure as Code (Pulumi, Ansible)
    \item Modernisierung bestehender DevContainer-Setups durch Entkopplung
    von Yocto-SDKs und signifikante Reduktion der Image-Größe
    \item DevEx-Consulting:
    Definition eines Platform-Engineering-Operating-Models
    zur Stärkung von Team-Autonomie und Self-Service-Fähigkeiten
\end{itemize}

\vspace{0.5em}

\subsection*{Platform Engineering \& Security, ARRI Cine Technik München}
\textit{Juli 2022 - Juni 2024}
\begin{itemize}[leftmargin=0.5cm]
    \item Security-Reviews interner Web-Applikationen:
    Schwachstellenanalyse (Authentication, Authorization, Input Validation),
    Erstellung von Empfehlungen zur Risikominimierung
    \item Pflege und Weiterentwicklung eines Yocto-basierten Embedded-Linux-Projekts
    inklusive LTS-Upgrade für professionelle Kamera-Systeme
    \item Migration von Jenkins zu GitLab CI
    mit Docker-basierten Runnern und reproduzierbaren Build-Pipelines
    \item Konzeption und Einführung einer Remote-Build- und Entwicklungsumgebung
    für Entwickler (Docker, VS Code Remote)
    \item Aufbau eines Self-Service-Plattformkonzepts mit Pulumi
    (u.\,a. GitLab- und Artifactory-Provisionierung)
    \item Entwicklung wiederverwendbarer CI-Module mit Dagger
    für KAS-Builds und Quality-Assurance-Workflows
    \item Infrastruktur-Management nach GitOps-Prinzipien
    unter Einsatz von Ansible
\end{itemize}

\vspace{0.5em}

\subsection*{DevOps Engineering, Philips Medizin Systeme Böblingen}
\textit{April 2021 - Juni 2022}
\begin{itemize}[leftmargin=0.5cm]
    \item Einführung von DevOps-Prinzipien im Umfeld
    des Model-Based Systems Engineering (MBSE)
    \item Aufbau einer GitHub-Actions-Pipeline
    zur Distribution von Java-Plugins
    \item Entwicklung eines CLI-Tools zur Benutzer- und Rechteverwaltung
    über die Cameo-REST-API
    \item Unterstützung verteilter Teams durch strukturierte
    Remote-Einarbeitung und Wissenstransfer
\end{itemize}


\section*{Berufserfahrung}

\subsection*{Philips Medizin Systeme Böblingen}
\textit{März 2006 - November 2019}

\textbf{DevOps Architect} (2017-2019)
\begin{itemize}[leftmargin=0.5cm]
    \item CI/CD-Pipeline für Yocto-basierte Embedded Builds
    \item Infrastructure as Code Einführung, Microservice-Architektur für Tooling
    \item DevOps-Konzept für regulierte Medizinproduktentwicklung
\end{itemize}

\textbf{DevOps Engineer / Scrum Master} (2014-2017)
\begin{itemize}[leftmargin=0.5cm]
    \item GitLab und Docker Registry Setup, Ansible-Provisionierung
    \item Scrum-Einführung als erster Scrum Master im Bereich
    \item Microservices in Go für Build-Tooling
\end{itemize}

\textbf{Embedded Developer} (2006-2014)
\begin{itemize}[leftmargin=0.5cm]
    \item Yocto-Einführung für Produktentwicklung
    \item C++ Realtime-Applikationen, Python-Tooling
    \item Build-System Entwicklung (Make, CMake)
\end{itemize}

\section*{Studium}
\textbf{Dipl. Ing. Softwaretechnik und Medieninformatik}, Hochschule Esslingen (2006)

\section*{IT-Security Zertifizierungen \& Aktivitäten}
\begin{itemize}[leftmargin=0.5cm]
    \item \textbf{OSCP} (Offensive Security Certified Professional) - bestanden im ersten Versuch (2020)
    \item \textbf{OSWE} (Offensive Security Web Expert) - in Bearbeitung (seit Februar 2026)
    \item \textbf{Web Application Security - SCP} (Skytale Certified Professional, 2020)
    \item \textbf{Bug Bounty}: HackerOne (2020-2021)
    \item \textbf{Responsible Disclosure}: u.\,a. Bundeswehr (mit offizieller Anerkennung)
\end{itemize}

\section*{Sprachen}
Deutsch (Muttersprache), Englisch (Fließend), Griechisch (Fließend)

\section*{Links}
\url{https://www.linkedin.com/in/charemma} | \url{https://charemma.de}

\end{document}
